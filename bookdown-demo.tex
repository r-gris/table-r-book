\documentclass[]{book}
\usepackage{lmodern}
\usepackage{amssymb,amsmath}
\usepackage{ifxetex,ifluatex}
\usepackage{fixltx2e} % provides \textsubscript
\ifnum 0\ifxetex 1\fi\ifluatex 1\fi=0 % if pdftex
  \usepackage[T1]{fontenc}
  \usepackage[utf8]{inputenc}
\else % if luatex or xelatex
  \ifxetex
    \usepackage{mathspec}
  \else
    \usepackage{fontspec}
  \fi
  \defaultfontfeatures{Ligatures=TeX,Scale=MatchLowercase}
\fi
% use upquote if available, for straight quotes in verbatim environments
\IfFileExists{upquote.sty}{\usepackage{upquote}}{}
% use microtype if available
\IfFileExists{microtype.sty}{%
\usepackage{microtype}
\UseMicrotypeSet[protrusion]{basicmath} % disable protrusion for tt fonts
}{}
\usepackage[margin=1in]{geometry}
\usepackage{hyperref}
\hypersetup{unicode=true,
            pdftitle={A Minimal Book Example},
            pdfauthor={Yihui Xie},
            pdfborder={0 0 0},
            breaklinks=true}
\urlstyle{same}  % don't use monospace font for urls
\usepackage{natbib}
\bibliographystyle{apalike}
\usepackage{color}
\usepackage{fancyvrb}
\newcommand{\VerbBar}{|}
\newcommand{\VERB}{\Verb[commandchars=\\\{\}]}
\DefineVerbatimEnvironment{Highlighting}{Verbatim}{commandchars=\\\{\}}
% Add ',fontsize=\small' for more characters per line
\usepackage{framed}
\definecolor{shadecolor}{RGB}{248,248,248}
\newenvironment{Shaded}{\begin{snugshade}}{\end{snugshade}}
\newcommand{\KeywordTok}[1]{\textcolor[rgb]{0.13,0.29,0.53}{\textbf{{#1}}}}
\newcommand{\DataTypeTok}[1]{\textcolor[rgb]{0.13,0.29,0.53}{{#1}}}
\newcommand{\DecValTok}[1]{\textcolor[rgb]{0.00,0.00,0.81}{{#1}}}
\newcommand{\BaseNTok}[1]{\textcolor[rgb]{0.00,0.00,0.81}{{#1}}}
\newcommand{\FloatTok}[1]{\textcolor[rgb]{0.00,0.00,0.81}{{#1}}}
\newcommand{\ConstantTok}[1]{\textcolor[rgb]{0.00,0.00,0.00}{{#1}}}
\newcommand{\CharTok}[1]{\textcolor[rgb]{0.31,0.60,0.02}{{#1}}}
\newcommand{\SpecialCharTok}[1]{\textcolor[rgb]{0.00,0.00,0.00}{{#1}}}
\newcommand{\StringTok}[1]{\textcolor[rgb]{0.31,0.60,0.02}{{#1}}}
\newcommand{\VerbatimStringTok}[1]{\textcolor[rgb]{0.31,0.60,0.02}{{#1}}}
\newcommand{\SpecialStringTok}[1]{\textcolor[rgb]{0.31,0.60,0.02}{{#1}}}
\newcommand{\ImportTok}[1]{{#1}}
\newcommand{\CommentTok}[1]{\textcolor[rgb]{0.56,0.35,0.01}{\textit{{#1}}}}
\newcommand{\DocumentationTok}[1]{\textcolor[rgb]{0.56,0.35,0.01}{\textbf{\textit{{#1}}}}}
\newcommand{\AnnotationTok}[1]{\textcolor[rgb]{0.56,0.35,0.01}{\textbf{\textit{{#1}}}}}
\newcommand{\CommentVarTok}[1]{\textcolor[rgb]{0.56,0.35,0.01}{\textbf{\textit{{#1}}}}}
\newcommand{\OtherTok}[1]{\textcolor[rgb]{0.56,0.35,0.01}{{#1}}}
\newcommand{\FunctionTok}[1]{\textcolor[rgb]{0.00,0.00,0.00}{{#1}}}
\newcommand{\VariableTok}[1]{\textcolor[rgb]{0.00,0.00,0.00}{{#1}}}
\newcommand{\ControlFlowTok}[1]{\textcolor[rgb]{0.13,0.29,0.53}{\textbf{{#1}}}}
\newcommand{\OperatorTok}[1]{\textcolor[rgb]{0.81,0.36,0.00}{\textbf{{#1}}}}
\newcommand{\BuiltInTok}[1]{{#1}}
\newcommand{\ExtensionTok}[1]{{#1}}
\newcommand{\PreprocessorTok}[1]{\textcolor[rgb]{0.56,0.35,0.01}{\textit{{#1}}}}
\newcommand{\AttributeTok}[1]{\textcolor[rgb]{0.77,0.63,0.00}{{#1}}}
\newcommand{\RegionMarkerTok}[1]{{#1}}
\newcommand{\InformationTok}[1]{\textcolor[rgb]{0.56,0.35,0.01}{\textbf{\textit{{#1}}}}}
\newcommand{\WarningTok}[1]{\textcolor[rgb]{0.56,0.35,0.01}{\textbf{\textit{{#1}}}}}
\newcommand{\AlertTok}[1]{\textcolor[rgb]{0.94,0.16,0.16}{{#1}}}
\newcommand{\ErrorTok}[1]{\textcolor[rgb]{0.64,0.00,0.00}{\textbf{{#1}}}}
\newcommand{\NormalTok}[1]{{#1}}
\usepackage{longtable,booktabs}
\usepackage{graphicx,grffile}
\makeatletter
\def\maxwidth{\ifdim\Gin@nat@width>\linewidth\linewidth\else\Gin@nat@width\fi}
\def\maxheight{\ifdim\Gin@nat@height>\textheight\textheight\else\Gin@nat@height\fi}
\makeatother
% Scale images if necessary, so that they will not overflow the page
% margins by default, and it is still possible to overwrite the defaults
% using explicit options in \includegraphics[width, height, ...]{}
\setkeys{Gin}{width=\maxwidth,height=\maxheight,keepaspectratio}
\IfFileExists{parskip.sty}{%
\usepackage{parskip}
}{% else
\setlength{\parindent}{0pt}
\setlength{\parskip}{6pt plus 2pt minus 1pt}
}
\setlength{\emergencystretch}{3em}  % prevent overfull lines
\providecommand{\tightlist}{%
  \setlength{\itemsep}{0pt}\setlength{\parskip}{0pt}}
\setcounter{secnumdepth}{5}
% Redefines (sub)paragraphs to behave more like sections
\ifx\paragraph\undefined\else
\let\oldparagraph\paragraph
\renewcommand{\paragraph}[1]{\oldparagraph{#1}\mbox{}}
\fi
\ifx\subparagraph\undefined\else
\let\oldsubparagraph\subparagraph
\renewcommand{\subparagraph}[1]{\oldsubparagraph{#1}\mbox{}}
\fi
\usepackage{booktabs}

%%% Use protect on footnotes to avoid problems with footnotes in titles
\let\rmarkdownfootnote\footnote%
\def\footnote{\protect\rmarkdownfootnote}

%%% Change title format to be more compact
\usepackage{titling}

% Create subtitle command for use in maketitle
\newcommand{\subtitle}[1]{
  \posttitle{
    \begin{center}\large#1\end{center}
    }
}

\setlength{\droptitle}{-2em}
  \title{A Minimal Book Example}
  \pretitle{\vspace{\droptitle}\centering\huge}
  \posttitle{\par}
  \author{Yihui Xie}
  \preauthor{\centering\large\emph}
  \postauthor{\par}
  \predate{\centering\large\emph}
  \postdate{\par}
  \date{2016-07-23}

\begin{document}
\maketitle

{
\setcounter{tocdepth}{1}
\tableofcontents
}
\chapter{Prerequisites}\label{prerequisites}

This is a \emph{sample} book written in \textbf{Markdown}. You can use
anything that Pandoc's Markdown supports, e.g., a math equation
\(a^2 + b^2 = c^2\).

For now, you have to install the development versions of
\textbf{bookdown} from Github:

\begin{Shaded}
\begin{Highlighting}[]
\NormalTok{devtools::}\KeywordTok{install_github}\NormalTok{(}\StringTok{"rstudio/bookdown"}\NormalTok{)}
\end{Highlighting}
\end{Shaded}

Remember each Rmd file contains one and only one chapter, and a chapter
is defined by the first-level heading \texttt{\#}.

To compile this example to PDF, you need to install XeLaTeX.

\chapter{Introduction}\label{intro}

Structures and data are appropriately complex and specialized in order
to ensure rigour and efficiency.

Spatial means more than maps in at least the sense that

\section{Terminology}\label{terminology}

Branch, piece/part, object, vertex, coordinate

object attribute metadata

\chapter{The GIS contract}\label{the-gis-contract}

GIS provides a table-based front-end where there is a one-to-one
relationship between a geometric object, and a row in a table that
contains attribute metadata about that object. I call this the ``GIS
contract'', and you can see this in the linked selections (brushing) in
QGIS, Manifold and other systems.

\chapter{\texorpdfstring{The \textbf{sp}
package}{The sp package}}\label{the-sp-package}

Spatial classes, formal inheritance, heirarchical objects.

The Spatial classes provided by the sp package are very widely used
because they provide the formal guarantee of the GIS contract, and use
this for a systematic coupling with other tools:

\begin{itemize}
\item
  the huge number of formats provided by GDAL input/output
\item
  powerful high-level methods for visualization, manipulation, analysis
  and modelling
\item
\end{itemize}

The attribute metadata are discrete but the geometry is continuous.
Spatial topology ties the geometry together, and database topology ties
the system together.

\chapter{\texorpdfstring{The \textbf{ggplot2}
package}{The ggplot2 package}}\label{the-ggplot2-package}

Data frames are the fundamental unit for \textbf{ggplot2}, but for
spatial data the first step is to decompose the hierarchical complexity
of a Spatial object to a single data frame.

This breaks the ``GIS contract'', since the individual objects are now
spread over multiple rows of the table of coordinates.

\chapter{Pros and cons}\label{pros-and-cons}

Ability to user-choose attributes from the data - independence of
analysis from the visualization

\section{Relations and the difference between sp and ggplot2
forms}\label{relations-and-the-difference-between-sp-and-ggplot2-forms}

There's a difficulty for non-experts to deal with relational data,
there's a level of abstraction in the process that provides confusion.
We see this in many fields, where a single table is the basic unit of
analysis and the lessons of database normalization are nowhere to be
seen. A common example is animal tracking data, which at the minimum
stores a trip (or burst or group) ID, x, y, date-time, and may include
individual or tag ID (object). Technically, these data should have a
metadata table with observations about the tag deployment (date,
location, recovery, animal departure/return date), an individual table
with observations on the animal species, and the table of coordinates of
the actual tag space-time measurements.

This is a well- recognized problem, especially in collaborative studies
where the entire data set is stored in a single CSV \ldots{} {[}probably
less ranting here,{]} but the relational table example is a good one.

A key definition here is the idea of a structural index (the row or
column number in a table or array) versus a relational index where the
value of a key is used to match records. The relational index can be
transferred from one data set to another by subsetting and appending,
and survive resorting generally - but the structural index cannot - it
must either be maintained in its position or be updated when the overal
dataset is subsetted, or changed.

Recurisve objects like lists in R stand in place of both structurual and
relational indexes, the structure of the list is an implicit marker of
the index - though it might also store a particular label.

\chapter{Examples}\label{examples}

\section{spbabel - two tables - round trip for sp and
ggplot2}\label{spbabel---two-tables---round-trip-for-sp-and-ggplot2}

The spbabel package provides a straightforward workflow for converting
from sp Spatial objects to a single table of all coordinates (analogous
to the fortify table), and back again.

This requires working with two tables, and can be considered as simply
starting with the sp object metadata table and then flattening out the
geometry of each piece of each object into a single table - this table
stores the x, y coordinates, a part identifier, a hole/island identifier
for parts, the order of the coordinates within a part, and the object
id. This could be a single table if all of the object metadata
attributes were copied onto all of the coordinates for each object - but
this is both wasteful and untidy in the sense that errors can be
introduced when a one-to-many relationship is duplicated across the
object ID and all the metadata values. Also may be inefficient (or not
given factor/character tricks, rle and so on).

Examples

sp to spbabel to ggplot2

ggplot2 to spbabel to sp

\section{rgl - structural indices}\label{rgl---structural-indices}

All rgl functions that plot linked coordinates use primitives (line
segments or triangles or quads) that are encoded as indexes into
coordinate arrays. There is no requirement that the coordinates be
unique, but they can be. Some rgl functions are use literally as an
index into 3-coords, and others use homogeneous coordinates with a 4th
coord (set this to one for the 3-coords behaviour).

Rgl includes an ear clipping triangulation algorithm so that polygons
can be converted to a surface composed of primitives. These surfaces are
much more general than GIS polygons or triangulations, since they can
``wrap -around''. A clear example is given in tetrahedron3d() and in
oh3d(). Ear clipping is fast, non-convex, and preserves input edges but
is otherwise not suited for choosing well-formed triangles.

The crux for the storage of objects in rgl is that each object is
standalone and there's no native set for storage of more than one
object.

Show creation of rgl objects from Spatial

creation of gris objects from Rvcg/rgl

\emph{This point is a general one in terms of the relational
hierarchies}.

\section{gris - four tables with vertex
topology}\label{gris---four-tables-with-vertex-topology}

\chapter{\texorpdfstring{Examples of applying the ``gris
framework''}{Examples of applying the gris framework}}\label{examples-of-applying-the-gris-framework}

Sp objects

sf objects

ggplot2 objects

rgl objects - difference when closed tetrahedron mesh or qmesh are used

\chapter{A shared framework for disparate spatial systems in
R?}\label{a-shared-framework-for-disparate-spatial-systems-in-r}

\section{Nesting}\label{nesting}

tidyr and gggeom show that nesting can be used to always work in a
single table, and sp can be emulated with single-nesting of the fortify
table, or more closely with double nesting, first on object and then on
piece. gggeom keeps attributes in separate lists, in order to allow for
different numbers of values in each (though tables can do that too . .
.).

The main issues with nesting into one table, with nested components, are
that: * it doesn't allow for many-to-one de-duplication indexing (unless
the nested component stores an index to another table - but that's not
one table). * it's not readily backed by a database

A system of normalized tables is ready for transfer to a database, and
can be read directly from a database \emph{without any specialist tools
for special types}

\section{Decomposition to relational
tables}\label{decomposition-to-relational-tables}

We have seen that a wide variety of data configurations can be converted
to a set of relational tables and that these give systematic and
straightforward pathways for reconstructing other forms. This provides
the opportunity for an API where specialized packages provide the
special methods to convert from and to the shared form, and so many
conversions then become automatic and easy.

Can we create simple idioms to encode decomposition from recursive
objects generally? The cascading semi-join makes for very simple
propagation of a subset from teh object table down through the other
tables, and a cascading inner join automatically builds the right
spbabel/fortify table that can be used directly, or as a stepping stone
to constructing recursive list forms.

The tidy initiative has show that these high-level processes can be
abstracted into commonly used tools, but so far it's been about tidying
up model outputs and reshaping between long and wide forms.

\chapter{Final Words}\label{final-words}

We have finished a nice book.

\bibliography{packages.bib,book.bib}


\end{document}
